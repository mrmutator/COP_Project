\section{Conclusion}\label{sec:conclusion}
We see that all models trained on utterance embeddings found with the techniques from \newcite{le2014distributed} outperform the trivial baseline of predicting the most frequent tag. This supports the hypothesis that such distributional representations can be used successfully for dialog act tagging.

However a simple model using a bag of words representation and a NBC outperforms these more intricate representations and models. \david{why is this? because of the short utterances? seems strange...}

Moreover we see that training the utterance embeddings with extra samples from the BNC does not have a big effect on the accuracy of the classifiers. This could be explained by the fact that the utterances in the BNC have been cleaned to be more like grammatically correct sentences, where the SwDA is in comprised of more accurately transcribed utterances. Therefor we recommend future work to use additional data that is more similar to the SwDA \david{i would like to say tag data or something like that but can't formulate this nicely} to investigate whether this increases the accuracy of the tagging.

When comparing the results of the classifiers on a coarser tag set we see that the accuracy dramatically increases for all classifiers. This shows that the classifiers make most mistakes confusing dialog acts that are closely related. \david{I don't know if this is true, but it would make sense, maybe support the claim with a confusion matrix?}

\david{Can somebody check this?}
A major problem with the work presented in this paper is the lack of naturally encorporating context into the representation of utterances. Using either the concatenation or addition of the previous utterance representation did not improve accuracy. Future work should therefore mainly concentrate on mending this weakness. One possible approach to this problem would be to use yet another deep learning technique called Long Short Term Memory (LSTM) as proposed by \newcite{lstm-original}. This recurrent neural network uses a so called memory cell with four connections, an input gate, an output gate, a self recurrent gate and a forget gate. The input gate can be used to alter the state of the memory cell, while the output gate can be used to affect the state of other layers. This technique can be applied to dialog act tagging by using the utterance embeddings presented earlier as inputs for the LSTM. The output of the LSTM cell can then be used to predict a tag for that utterance. Since the utterances are fed to the LSTM in sequence, the memory cell can be thought to represent the contextual knowledge of the conversation up to that point.

\david{are we missing anything?}
\david{some general conclusion to wrap it all up here.}