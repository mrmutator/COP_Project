\section{Related Work}\label{sec:relwork}
Several approaches have been proposed for classifying dialog acts. Most of them rely on supervised trained models and use hand-crafted features extracted for all utterances. Some recent work shows that using distributional representations for dialog act classification outperforms these methods. We briefly present some of the most relevant work in this section.

The authors of \newcite{stolcke2000} predict dialog acts by modeling a conversation as a Hidden Markov Model (HMM). A sequence of dialog acts is represented as a \emph{discourse model} where the probability of the next dialog act depends on the \emph{n} previous dialog acts. They integrate this model with a \emph{language model} for each separate dialog act, which computes the probability for a certain dialog act tag given the occurrence of all \emph{word n-grams} in that utterance. \newcite{stolcke2000} also train models on the actual speech signals, where the `language model' is trained on prosodic and acoustic evidence. When considering the models trained on the dialog transcripts we can see that in this an utterance is represented as a bag of n-grams. This might not capture the entire composition of that utterance. We aim to train a single model to directly find the dialog act of an utterance given some representation of that utterance.
%We will try to find a representation that captures the composition of an utterance in a better way

In \newcite{kalchbrenner} a Recurrent Convolutional Neural Network (RCNN) is trained in a supervised manner on a dialog corpus, which achieves state of the art results on the dialog act tagging task. The RCNN learns an integrated \emph{discourse model} and a \emph{sentence model} from a specific corpus, where the utterance representation is derived from individual word vectors, which are chosen randomly. We feel that this representation can become a lot richer if it is learned as in \newcite{le2014distributed}, where word vectors have some distributional meaning. Another strength of the approach of \newcite{le2014distributed} is that it is possible to train these utterance embeddings in an unsupervised manner, making it possible to include additional, possibly unannotated, corpora. 

An investigation on the contribution of distributional semantic information to the dialog act tagging task was conducted in \newcite{milajevs}. It was found that such information did improve tagging when compared to simple bag-of-words approaches. However only very simple distributional representations were investigated in this work. Words were represented as vectors of their co-occurrence counts. The space of these vectors was reduced using Singular Value Decomposition (SVD) to obtain a denser representation. Utterances were then represented as point-wise multiplications or additions of these vectors, which implicates the loss of any compositional information. The work of \newcite{milajevs} was not able to outperform the earlier work on dialog act tagging presented before. We believe that with richer representations of utterances, where we also incorporate composition, we can improve on this performance.

%In our work we will use the same \emph{discourse model} as \newcite{stolcke2000} and represent the entire dialog as an HMM. However for our \emph{sentence model} we will train different classifiers based on the embeddings, which are learned using the techniques from \newcite{le2014distributed}. We will explain how we construct utterance embeddings in the next section. In section \ref{sec:method} we explain how we use these representations to build a \emph{sentence model} as well and briefly repeat the \emph{discourse model} as an HMM.

We will attempt to find a model that directly models the probability of dialog tags given a representation of an utterance. Unlike \newcite{stolcke2000} who learns a separate model for the dialog and for each dialog act. We will adopt techniques proposed by \newcite{le2014distributed}, which we will explain in the next section, to capture the words and composition of an utterance. Just like \newcite{milajevs} we will use these rich representations to train classifiers. Note that we will only model context by concatenating utterance representations and leave a more informative approach to modeling context for future work.
